\documentclass[11pt,letterpaper]{article}
%\documentclass[11pt,a4paper]{report}

\usepackage{amssymb,amsmath,amsthm} 
\usepackage[margin=2cm]{geometry}
\usepackage{fancyhdr}
\usepackage{enumitem}
\usepackage[compact]{titlesec}
\usepackage{graphicx,ctable,booktabs,subcaption}

\usepackage{xparse,hyperref,parskip}

%\newcommand{\abs}[1]{\left|#1\right|}

\newcommand{\semester}{Spring 2022}
\newcommand{\due}{Tuesday, March 29}

\newcommand{\bigo}{\mathcal{O}}

\pagestyle{fancy}
\lhead{ }
\chead{\footnotesize Math 3338\quad  Numerical Methods\quad  \semester}
\rhead{\footnotesize \thepage}
\setlength{\parindent}{0cm}
\setlist{noitemsep}



\newtheorem{theorem}{Theorem}

\input{defs.tex}

%Defines the problem environment with arguments Points and Solution gap
\input{problem_env.tex}



\begin{document}

\begin{center}
{\huge{\bf  Numerical Methods}} \\[1.5ex]
{\bf Math 3338 -- \semester}\\[1.5ex]
{\Large{\bf Homework 19\ \\[2ex] Graph Eigenvectors}}\\
\end{center}
\vspace{2mm}








In Problems \ref{lap:start} -- \ref{lap:end} we're going to explore the Laplacian matrix $L$.
This is defined as $L = D-A$ where $D$ is the diagonal matrix with entries given by the degree
of each vertex and $A$ is the adjacency matrix.



\begin{problem}
\label{lap:start}
 This is not a computer question. Verify that $\vec{1}$ is an eigenvector of the Lapacian matrix
with eigenvalue $0$. This is a proof you'll type in \LaTeX.
\end{problem}

\begin{problem}
 Here is another proof. Suppose $G$ has $k$ connected components. Prove the eigenvalue $0$ has
multiplicity $k$ in the Lapacian.
\end{problem}

\begin{problem}
The Lagrangian is a positive semidefinite matrix. This means all the eigenvalues should be greater
than or equal to zero. Verify this (you don't need to prove it, just computationally).
\end{problem}


\begin{problem}
Drawing graphs is extremely challenging. As long as the vertices and edges are correct, you can draw
the graph however you like. Luckily, the eigenvectors of the Laplacian can help. Take the eigenvectors
of the two smallest non-zero eigenvalues and use the coordinates as the $(x,y)$ values of the 
vertices. This embedding is ``minimal'' in some sense. Write a program to draw graphs in this manner. 
\end{problem}

\begin{problem}
\label{lap:end}

The file \texttt{all\_graphs.dat} contains a database of all graphs with no isolated vertices 
with less than or equal to $6$ vertices. The file \texttt{all\_graphs.pdf} is a drawing of each
graph.

Use the program from the previous problem to redraw all the graphs. You should automate the creation
of the \LaTeX\ file. (I did). \texttt{matplotlib.pyplot} will be useful here. Essentially you'll be placing points and lines.
\end{problem}











\end{document}





































