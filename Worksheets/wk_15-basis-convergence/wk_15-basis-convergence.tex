\documentclass[11pt,letterpaper]{article}
%\documentclass[11pt,a4paper]{report}

\usepackage{amssymb,amsmath,amsthm} 
\usepackage[margin=2cm]{geometry}
\usepackage{fancyhdr}
\usepackage{enumitem}
\usepackage[compact]{titlesec}
\usepackage{graphicx,ctable,booktabs,subfigure}

\usepackage{xparse,hyperref,parskip}

%\newcommand{\abs}[1]{\left|#1\right|}

\newcommand{\semester}{Spring 2022}
\newcommand{\due}{Tuesday, March 8}


\pagestyle{fancy}
\lhead{ }
\chead{\footnotesize Math 3338\quad  Numerical Methods\quad  \semester}
\rhead{\footnotesize \thepage}
\setlength{\parindent}{0cm}
\setlist{noitemsep}



\input{defs.tex}

%Defines the problem environment with arguments Points and Solution gap
\input{problem_env.tex}



\begin{document}

\begin{center}
{\huge{\bf  Numerical Methods}} \\[1.5ex]
{\bf Math 3338 -- \semester}\\[1.5ex]
{\Large{\bf Homework 15 (Due: \due)\ \\[2ex] Basins of Convergence}}\\
\end{center}
\vspace{2mm}

%\section{Reading}

%\begin{table}[!ht]
% \centering
% \begin{tabular}{ll}
%   CP & 5.2, 5.3, 5.4 \\
% NMEP & 6.3
% \end{tabular}
%\caption{Sections Covered}
%\end{table}

\begin{problem}
 Did you know Newton's method works for complex zeros as well? Try it, set $f(x)=x^5-4x^3+2x-1$
and run Newton's method with $x_0=1j$ (this is the $i$ in Python). For this problem we are going 
to enhance the images we made last time, this time we'll work over the complex plane. To do
this, do the following.
\begin{enumerate}
 \item Since we're dealing with floats, create a \texttt{fuzzy\_equal} function that returns True
if the difference between two numbers is ``small'' ($<10^{-6}$) and False otherwise.

 \item Create a function \texttt{find\_zeros} that will... find the zeros of a function. To do
this search a $10\times 10$ grid from $(a,b)\times (a,b)$, these are complex numbers. You should
return a list containing the zeros. In general, this may not find all the zeros. 

\item Create another function \texttt{find\_basins}. This funtion will return an $N\times N$ grid
of integers which correspond to the index of that root in the zero list. To create this, I recommend
starting with an $N\times N$ grid of complex numbers, using those to find the convergence root,
and then finding the index in the list.

\item Use \texttt{plt.imshow} to plot the result of \texttt{find\_basins}. Save your figure as a 
PDF. You'll submit it on Canvas.

\end{enumerate}
\end{problem}


\begin{problem}
 Use the previous problem to create a graph of the convergence of your ID polynomial. Suppose your
ID number is 1234567, your ID polynomial is,
\[
 f(x) = x^6+2x^5+3x^4+4x^3+5x^2+6x+7.
\]
\end{problem}







\end{document}




































