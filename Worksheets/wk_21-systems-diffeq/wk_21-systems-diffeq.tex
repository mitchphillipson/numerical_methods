\documentclass[11pt,letterpaper]{article}
%\documentclass[11pt,a4paper]{report}

\usepackage{amssymb,amsmath,amsthm} 
\usepackage[margin=2cm]{geometry}
\usepackage{fancyhdr}
\usepackage{enumitem}
\usepackage[compact]{titlesec}
\usepackage{graphicx,ctable,booktabs,subcaption}

\usepackage{xparse,hyperref,parskip}

%\newcommand{\abs}[1]{\left|#1\right|}

\newcommand{\semester}{Spring 2022}
\newcommand{\due}{Tuesday, April 5}


\pagestyle{fancy}
\lhead{ }
\chead{\footnotesize Math 3338\quad  Numerical Methods\quad  \semester}
\rhead{\footnotesize \thepage}
\setlength{\parindent}{0cm}
\setlist{noitemsep}



\newtheorem{theorem}{Theorem}

\input{defs.tex}

%Defines the problem environment with arguments Points and Solution gap
\input{problem_env.tex}



\begin{document}

\begin{center}
{\huge{\bf  Numerical Methods}} \\[1.5ex]
{\bf Math 3338 -- \semester}\\[1.5ex]
{\Large{\bf Worksheet 21\ \\[2ex] Systems of Differential Equations}}\\
\end{center}
\vspace{2mm}

\section{Reading}

\begin{table}[!ht]
 \centering
 \begin{tabular}{ll}
   CP &  8.1, 8.2 \\
 NMEP &  Chapter 7
 \end{tabular}
\caption{Sections Covered}
\end{table}

\section{Overview}
Believe it or not, not everything depends on only one variable. In these problems it's going 
to be convenient to consider our independent variable as time, $t$. Suppose we have functions 
$x_i(t)$ for $1\le i \le n$ and relations,
\[
 \frac{dx_i}{dt} = f_i(t,x_1,\dots,x_n).
\]
This creates a system of differential equations. We want to solve this system. 

To solve this, we are going to modify this into a vector equation. Let $\vec{x}(t) = (x_1(t),\dots,x_n(t))$
and $\vec{f} = (f_1,\dots,f_n)$. Then, our system becomes
\[
 \frac{d\vec{x}}{dt} = f(t,\vec{x}).
\]
Writing like this allows us to vectorize Runge-Kutta,
\begin{align*}
        \vec{k}_1 &= hf(t,\vec{x}) \\ 
        \vec{k}_2 &= hf(t+.5h,\vec{x}+.5\vec{k}_1) \\
        \vec{k}_3 &= hf(t+.5h,\vec{x}+.5\vec{k}_2) \\
        \vec{k}_4 &= hf(t+h,\vec{x}+\vec{k}_3) \\
        \vec{x}_{new} &= \vec{x}+1/6(\vec{k}_1+2\vec{k}_2+2\vec{k}_3+\vec{k}_4).
\end{align*}


In other words, multidimensional Runge-Kutta is just Runge-Kutta in each dimension.

\newpage

\begin{center}
{\huge{\bf  Numerical Methods}} \\[1.5ex]
{\bf Math 3338 -- \semester}\\[1.5ex]
{\Large{\bf Homework 21 (Due: \due)}}\\
\end{center}
\vspace{2mm}


\begin{problem}
 Modify your \texttt{runge\_kutta} method from last time to handle multidimensional array inputs. 
In other words, \texttt{runge\_kutta} should accept a function $f$ that returns an array, and the
input $y_0$ should be an array.

This function should use vectorization \emph{and should be identical (or very similar) to your
original function}. For example, when I wrote mine I changed two small things. And that's it.
\end{problem}



\begin{problem}
 The Lotka-Volterra equations are a mathematical model of predator-prey interactions between
biological species. Let two variables $x$ and $y$ be proportional to the size of populations of
two species, traditionally ``rabbits'' and ``foxes''. 

In the Lotka-Volterra model the rabbits reproduce at a rate proportional to their population, but
are eaten by foxes at a rate proportional to their own population and the population of foxes:
\[
\frac{dx}{dt} = \alpha x - \beta xy,
\]
where $\alpha$ and $\beta$ are constants. At the same time foxes reproduce at a rate proportional to the 
rate at which they eat rabbits but also die of old age at a rate proportional to their own 
population:
\[
 \frac{dy}{dt} = \gamma xy - \delta y,
\]
where $\gamma$ and $\delta$ are constants.
\begin{enumerate}
 \item Use the Runge-Kutta method for the case $\alpha=1$, $\beta=\gamma=.5$, and $\delta = 2$,
starting with the initial condition $x=y=2$. Make a graph showing both $x$ and $y$ as a function
of time on the same axes from $t=0$ to $t=30$. Be sure to label the graph.
 \item Describe in words what is going on in the system, in terms of rabbits and foxes.
\end{enumerate}

\end{problem}


\begin{problem}
The Lorenz equations are:
\begin{align*}
 \frac{dx}{dt} &= \sigma(y-x), & \frac{dy}{dt}&= rx-y-xz, & \frac{dz}{dt}&=xy-bz
\end{align*}
where $\sigma$, $r$ and $b$ are constants. 

These equations were first studied by Edward Lorenz in 1963, who derived them from a simplified 
model of weather patterns. The reason for their fame is that they wer one of the first good examples
of \emph{deterministic chaos}, the occurrence of apparently random motion even though there is
no randomness. 

Solve the Lorenz equations for the case $\sigma=10$, $r=28$, and $b=\frac{8}{3}$ in the 
range $t=0$ to $t=50$ with initial conditions $(x,y,z) = (0,1,0)$. Use \emph{at minimum}
10,000 steps. 

\begin{enumerate}
 \item Make a plot of $y$ vs $t$. Note the unpredictable nature of the motion.
 \item Make three more plots $z$ vs $x$, $y$ vs $x$ and $z$ vs $y$. Ensure the figsize of each
is large so you can see detail.
\end{enumerate}

Include all these in a PDF.


\end{problem}





\end{document}






































