\documentclass[11pt,letterpaper]{article}
%\documentclass[11pt,a4paper]{report}

\usepackage{amssymb,amsmath,amsthm} 
\usepackage[margin=2cm]{geometry}
\usepackage{fancyhdr}
\usepackage{enumitem}
\usepackage[compact]{titlesec}
\usepackage{graphicx,ctable,booktabs,subfigure}

\usepackage{xparse,hyperref,parskip}

%\newcommand{\abs}[1]{\left|#1\right|}

\newcommand{\semester}{Spring 2022}
\newcommand{\due}{Thursday, February 24}


\pagestyle{fancy}
\lhead{ }
\chead{\footnotesize Math 3338\quad  Numerical Methods\quad  \semester}
\rhead{\footnotesize \thepage}
\setlength{\parindent}{0cm}
\setlist{noitemsep}



\newtheorem{theorem}{Theorem}

\input{defs.tex}

%Defines the problem environment with arguments Points and Solution gap
\input{problem_env.tex}



\begin{document}

\begin{center}
{\huge{\bf  Numerical Methods}} \\[1.5ex]
{\bf Math 3338 -- \semester}\\[1.5ex]
{\Large{\bf Worksheet 12\ \\[2ex] LU-Decomposition}}\\
\end{center}
\vspace{2mm}


\section{Reading}

\begin{table}[!ht]
 \centering
 \begin{tabular}{ll}
   CP &  6.1 \\
 NMEP & 2.1, 2.2, 2.3, 2.5
 \end{tabular}
\caption{Sections Covered}
\end{table}

\section{$LU$-decomposition}
As you should have already seen, solving a system that is triangular is quite easy. That's the basic
idea behind $LU$-decomposition, write $A = LU$ where $L$ is lower triangular and $U$ is upper 
triangular. The idea is that if $Ax=b$ and $A = LU$, then $LUx=b$, so we solve $Ly=b$ then $Ux = y$. 

How do we find $L$ and $U$? Suppose
\[
 A = \begin{bmatrix}
a_{00} & a_{01} & a_{02} \\
a_{10} & a_{11} & a_{12} \\
a_{20} & a_{21} & a_{22} 
\end{bmatrix}
\]
Think about Gaussian elimination in terms of matrix multiplication. The first step of Gaussian
elimination is to make the top left be 1, and make 0's below it. If
\[
 T_1 = \frac{1}{a_{00}}\begin{bmatrix}
1 & 0 & 0 \\
-a_{10} & a_{00} & 0 \\
-a_{20} & 0 & a_{00}
\end{bmatrix}
\]
then,
\[
T_1A = \begin{bmatrix}
1 & b_{01} & b_{02} \\
0 & b_{11} & b_{12} \\
0 & b_{21} & b_{22} \\
\end{bmatrix}
\]
The trick here was finding $T_1$. Do the multiplication $T_1A$ on a piece of paper to see exactly what is happening. Try to justify to yourself where ``Gaussian Elimination'' comes into play.

Do this two more times to find,
\begin{align*}
T_2 &= \frac{1}{b_{11}}\begin{bmatrix}
b_{11} & 0 & 0 \\
0 & 1 & 0 \\
0 & -b_{21} & b_{11}
\end{bmatrix}
&
T_3 &= \frac{1}{c_{22}}\begin{bmatrix}
c_{22} & 0 & 0 \\
0 & c_{22} & 0 \\
0 & 0 & 1
\end{bmatrix}
\end{align*}

Combining these we see,
\[
T_3T_2T_1A = U
\]
where $U$ is upper triangular. Set $L = T^{-1}$, or $L_i = T_i^{-1}$, and we finally have
\[
 A = LU
\]
where $L$ is lower triangular.

It's easy to see,
\begin{align*}
T_1^{-1} &= \begin{bmatrix}
a_{00} & 0 & 0 \\
a_{10} & 1 & 0 \\
a_{20} & 0 & 1
\end{bmatrix}
&
T_2^{-1} &= \begin{bmatrix}
1 & 0 & 0 \\
0 & b_{11} & 0 \\
0 & b_{21} & 1
\end{bmatrix}
&
T_3^{-1} &= \begin{bmatrix}
1 & 0 & 0 \\
0 & 1 & 0 \\
0 & 0 & c_{22}
\end{bmatrix}
\end{align*}




\newpage

\begin{center}
{\huge{\bf  Numerical Methods}} \\[1.5ex]
{\bf Math 3338 -- \semester}\\[1.5ex]
{\Large{\bf Homework 12 (Due: \due)}}\\
\end{center}
\vspace{2mm}


\begin{problem}
 Write a function called \texttt{forwardsub(A,b)} that takes an lower-triangular matrix $A$ and
a column vector $b$ and returns a vector $x$ so that $Ax=b$. 
\end{problem}



\begin{problem}
\label{prob:lu}
  Write a program called \texttt{lu\_decomp} that finds the $LU$-decomposition of a matrix $A$. Your input should be \texttt{A} where $A$ is a matrix (as a nested array) and $b$ is a column array. Return a tuple \texttt{L,U}.
\end{problem}

\begin{problem}
 Write a program called \texttt{lu\_solve} that solves a system $Ax=b$ using $LU$-decomposition. You input should be \texttt{L,U,b} where $Ax=b$ and $A=LU$. The output is a column vector \texttt{x}.
\end{problem}



\begin{problem}
$LU$-decomposition is a lot of work, it's computationally intensive. Why do we do this as opposed to Gaussian Elimination? The answer is when you need to solve a system more than once. You can cache (or save) the $LU$-decomposition and quickly solve follow up systems of equations. Let's test this. The file ``LU.pickle'' on Canvas contains a list of tuples of the form \texttt{[(A,[b])]}, where $[b]$ is a list of solutions. For each $A$, time how long it takes to 
\begin{enumerate}
 \item Use Gaussian Elimination to solve $Ax=b$ for each b
 \item Find $LU$-decomposition and use that to solve $LUx=b$ for each b\footnote{you should only find the $LU$-decomp once}.
\end{enumerate}
Make a table to compare the evaluation times, include a column with the number of $b$'s for each $A$. Write a little paragraph discussing the results.\footnote{I didn't see any speed improvements. Do you?} 
\end{problem}




\end{document}




































