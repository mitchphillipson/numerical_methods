\documentclass[11pt,letterpaper]{article}
%\documentclass[11pt,a4paper]{report}

\usepackage{amssymb,amsmath,amsthm} 
\usepackage[margin=2cm]{geometry}
\usepackage{fancyhdr}
\usepackage{enumitem}
\usepackage[compact]{titlesec}
\usepackage{graphicx,ctable,booktabs,subfigure}

\usepackage{xparse,hyperref,parskip}

%\newcommand{\abs}[1]{\left|#1\right|}

\newcommand{\semester}{Spring 2022}
\newcommand{\due}{Thursday, February 10}


\pagestyle{fancy}
\lhead{ }
\chead{\footnotesize Math 3338\quad  Numerical Methods\quad  \semester}
\rhead{\footnotesize \thepage}
\setlength{\parindent}{0cm}
\setlist{noitemsep}



\input{defs.tex}

%Defines the problem environment with arguments Points and Solution gap
\input{problem_env.tex}



\begin{document}

\begin{center}
{\huge{\bf  Numerical Methods}} \\[1.5ex]
{\bf Math 3338 -- \semester}\\[1.5ex]
{\Large{\bf Homework 8 (Due: \due)\ \\[2ex] Applications of Integration}}\\
\end{center}
\vspace{2mm}

%\section{Reading}

%\begin{table}[!ht]
% \centering
% \begin{tabular}{ll}
%   CP & 5.2, 5.3, 5.4 \\
% NMEP & 6.3
% \end{tabular}
%\caption{Sections Covered}
%\end{table}


Create a file called ``name\_integrals.py'', where name is your name (or initials or something short).
In this file put the functions \texttt{latex\_table}, \texttt{trapezoidal}, \texttt{trap\_error},
and \texttt{romberg}. Modify the last two to only return the value of the integral (not the number
of steps).


\begin{problem}
 Not all integrals are over a finite domain. For example, our methods will fail if we want to evaluate
$\int_0^{\infty} f(x)\,dx$. This is solved using a change of variables. We want a function, $F(x)$,
defined on the interval $(a,b)$ so that $\lim_{n\rightarrow a} F(x) = 0$ and 
$\lim_{n\rightarrow b} F(x) = \infty$. One common function is,
\[
F(x) = \frac{x}{1-x}
\]
\end{problem}
on the interval $(0,1)$. The transformation is then,
\[
\int_0^1 f(F(t))F'(t)\,dt = \int_0^1 f\left(\frac{t}{1-t}\right)\cdot\frac{1}{(t-1)^2}\,dt.
\]
Another function, that may be better, is,
\[
F(x) = \tan(x).
\]
This has the benefit of being able to transform the domain $(-\infty,\infty)$.

The \emph{normal distribution} is given by,
\[
f(x) = \frac{1}{\sigma\sqrt{2\pi}}e^{-\frac{1}{2}\left(\frac{x-\mu}{\sigma}\right)^2}.
\]
Where $\mu$ is the mean and $\sigma$ is the standard deviation. This is typically standardized
so that $\mu=0$ and $\sigma=1$. We'll assume $f(x)$ is the standardized curve.

\begin{enumerate}
 \item Verify that $\displaystyle\int_{-\infty}^\infty f(x) = 1$.
 \item In statistics it's important to be able to evaluate $\int_{-\infty}^a f(x)\,dx$. To do this
they use a table, see Canvas for an example (normal.pdf). This table shows all values of that
integral for $-3.49\le a\le 3.49$ in steps of $.01$. Remake this table accurate to 6 decimals.
\end{enumerate}




\begin{problem}
 the Gamma function is a highly important function in Mathematics, Physics, Chemistry, and 
Statistics. The Gamma function is defined for $x>0$,
\[
 \Gamma(x) = \int_0^\infty t^{x-1}e^{-t}\,dt
\]

\begin{enumerate}
 \item Write a function to evaluate $\Gamma(x)$.
 \item Make a table evaluating $\Gamma(x)$ for integers $i=1,2,\dots,30$. 
 \item You should recognize the numbers you just computed. What are they? You may need to 
increase your tolerance to see this.
 \item Plot both $\Gamma(x)$ and $\Gamma(x)+\sin(\pi x)$ on the same axes. Save this figure 
and include it in the PDF. You can see they intersect, where do you think they intersect?
\end{enumerate}



\end{problem}





\end{document}




































