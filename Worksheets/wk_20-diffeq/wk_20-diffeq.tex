\documentclass[11pt,letterpaper]{article}
%\documentclass[11pt,a4paper]{report}

\usepackage{amssymb,amsmath,amsthm} 
\usepackage[margin=2cm]{geometry}
\usepackage{fancyhdr}
\usepackage{enumitem}
\usepackage[compact]{titlesec}
\usepackage{graphicx,ctable,booktabs,subfigure}

\usepackage{xparse,hyperref,parskip}

%\newcommand{\abs}[1]{\left|#1\right|}

\newcommand{\semester}{Spring 2022}
\newcommand{\due}{Thursday, March 31}


\pagestyle{fancy}
\lhead{ }
\chead{\footnotesize Math 3338\quad  Numerical Methods\quad  \semester}
\rhead{\footnotesize \thepage}
\setlength{\parindent}{0cm}
\setlist{noitemsep}

\newcommand{\bigo}{\mathcal{O}}

\newtheorem{theorem}{Theorem}

\input{defs.tex}

%Defines the problem environment with arguments Points and Solution gap
\input{problem_env.tex}



\begin{document}

\begin{center}
{\huge{\bf  Numerical Methods}} \\[1.5ex]
{\bf Math 3338 -- \semester}\\[1.5ex]
{\Large{\bf Worksheet 20\ \\[2ex] First Order Differential Equations}}\\
\end{center}
\vspace{2mm}


\section{Reading}

\begin{table}[!ht]
 \centering
 \begin{tabular}{ll}
   CP &  8.1, 8.2 \\
 NMEP &  Chapter 7
 \end{tabular}
\caption{Sections Covered}
\end{table}

\section{Overview}
Suppose $y$ is a function of $x$. We want to solve equations of the form,
\[
\frac{dy}{dx} = f(x,y).
\]

\section{Euler's Method}
This is the simplest numerical method. Here is the process,
\begin{enumerate}
 \item Start with a point $(x_0,y_0)$ and a step size $h$
 \item $f(x_{0},y_{0})$ is the slope at the point $(x_{0},y_{0})$, make the tangent line of the
function at this point $y=f(x_0,y_0)(x-x_0)+y_0$.
 \item Set $x_1 = x_0+h$ and $y_1 = f(x_0,y_0)(x_1-x_0)+y_0$.
 \item Continue until you hit the upper limit of $x$-values.
\end{enumerate}

This is an approximation technique. The smaller the step size, the better the approximation. It's
error term is $\bigo(h^2)$ (this is derived in the book). However, this is a tricky bit here
\emph{each step} is accurate to $\bigo(h)$ and we are potential computing a ton of steps. The error
will compound, you'll see this in the homework. 


\section{Runge-Kutta}
We've already learned the first order Runge-Kutta method! It's the same as Euler's method. It's 
called ``first order" because it is accurate to $\bigo(h)$ (in other words the error term is
$\bigo(h^2)$).

\subsection{Second order Runge-Kutta}

This is accurate to $\bigo(h^2)$ with error terms $\bigo(h^3)$. The book has a full explanation
for how this works. Very similar to Euler's method, this is an iterative technique except instead
of approximating the slope at $x$, we approximate the slope at $x+\frac{1}{2}h$. 

A single iteration is given by the calculations,
\begin{align*}
 k_1 &= hf(x,y) \\ 
 k_2 &= hf(x+\frac{1}{2}h,y+\frac{1}{2}k_1)  \\
 y_{new} &= y + k_2 
\end{align*}
where $f(x,y) = \frac{dy}{dx}$.


\subsection{Fourth order Runge-Kutta}
This is more complicated than the second order, but also far more complicated. It's accurate to 
$\bigo(h^4)$, which is amazing.
\begin{align*}
        k_1 &= hf(x,y) \\ 
        k_2 &= hf(x+.5h,y+.5k_1) \\
        k_3 &= hf(x+.5h,y+.5k_2) \\
        k_4 &= hf(x+h,y+k_3) \\
        y_{new} &= y+1/6(k_1+2k_2+2k_3+k_4)
\end{align*}


\section{Vector Fields}
A vector field is a field of vectors. It tells you which direction of the slope at each point in
the plane. These are easy to make in Python,
\begin{verbatim}
    X,Y = np.meshgrid(np.linspace(xmin,xmax,Nx),np.linspace(ymin,ymax,Ny))
    plt.quiver(X,Y,1,f(X,Y),color="lightgrey")
\end{verbatim}




\section{Mean Absolute Error}
Suppose you have to vectors $x$ and $y$. You can determine how ``close'' they are using the mean
absolute error,
\[
 \dfrac{ \sum_{i=1}^n \abs{x_i-y_i}}{n}
\]
This is a stats thing, feel free to Google it. It tends to be used quite often in Machine learning 
to as a measure of how close predictions are to data.







\newpage

\begin{center}
{\huge{\bf  Numerical Methods}} \\[1.5ex]
{\bf Math 3338 -- \semester}\\[1.5ex]
{\Large{\bf Homework 20 (Due: \due)}}\\
\end{center}
\vspace{2mm}


\begin{problem}
 Create three functions, \texttt{euler}, \texttt{runge\_kutta2}, and \texttt{runge\_kutta}. The
inputs to each of these should be $(f,x,y_0)$ where $f$ represents $\frac{dy}{dx}$ so
$f$ is a bivariate function, $x$ is an
array containing $x$-values (this will include the step size, $h$) and $y_0$ is the initial
$y$-value. They should return an array of $y$-values.
\end{problem}




\begin{problem}
 Consider the differential equation,
\[
 \frac{dy}{dx} = xy
\]
This is separable differential equation, so we can actually solve this. The solution is 
$y=A\cdot\exp\left(\frac{x^2}{2}\right)$. Plot the following on a single graph,

\begin{enumerate}
 \item Create a vector field of $\frac{dy}{dx}$ for $-2\le x,y\le 2$. Make the color of the vector field light grey.
 \item Plot $f(x) = A\cdot\exp\left(\frac{x^2}{2}\right)$ for $-2\le x\le 2$. You'll need to solve
for $A$. Label this curve ``exact''
 \item Plot three more curves, one for euler, runge\_kutta2 and runga\_kutta. Make the $x$-vector
start at $-2$, end at $2$ with 10 equally spaced points between (in other words, a linspace). Label 
these appropriately.
 \item Include a legend.
\end{enumerate}

Include this graph in a PDF along with a visual description of what you see occurring. Note, a visual
description uses words. Calculate the mean absolute error of each as well and make a table.

\end{problem}



\begin{problem}
 Use the same differential equation from the previous problem. Put the following on a single graph
\begin{enumerate}
 \item Create a vector field of $\frac{dy}{dx}$ for $-2\le x,y\le 2$. Make the color of the vector field light grey.
 \item Plot $f(x) = A\cdot\exp\left(\frac{x^2}{2}\right)$ for $-2\le x\le 2$. You'll need to solve
for $A$. Label this curve ``exact''
 \item Plot an approximation using Euler's method with 1,000 steps.
 \item Plot an approximation using Runge\_Kutta2 with 100 steps.
 \item Plot an approximation using Runge\_Kutta method with 10 steps.
 \item Include a legend.
\end{enumerate}

Include this graph in a PDF along with a visual description of what you see occurring. Note, a visual
description uses words. Calculate the mean absolute error of each as well and make a table.


\end{problem}


\begin{problem}
 Use the same differential equation from the previous problem. Put the following on a single graph
\begin{enumerate}
 \item Create a vector field of $\frac{dy}{dx}$ for $-2\le x,y\le 2$. Make the color of the vector field light grey.
 \item Plot $f(x) = A\cdot\exp\left(\frac{x^2}{2}\right)$ for $-2\le x\le 2$. You'll need to solve
for $A$. Label this curve ``exact''
 \item Plot an approximation using Euler's method with 10,000 steps.
 \item Plot an approximation using Euler's method with 100 steps.
 \item Plot an approximation using Euler's method with 10 steps.
 \item Include a legend.
\end{enumerate}

Include this graph in a PDF along with a visual description of what you see occurring. Note, a visual
description uses words. Calculate the mean absolute error of each as well and make a table.


\end{problem}

\end{document}



































